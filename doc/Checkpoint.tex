\documentclass{article}
\usepackage[utf8]{inputenc}
\usepackage{geometry}

\title{ARM11 - Checkpoint Report}
\author{Group 47 - \\
Samuel Andresen, Claire Shen, Christopher Kuan, \\
Jordan Emery, Sanidhya Singh}
\date{May/June 2020}

\begin{document}

\maketitle

\section*{Work Distribution and Co-ordination}
Communication was done via a WhatsApp group for small queries and updates and 
the Microsoft Teams group for meetings. While only Samuel and Claire (Siran) 
attended the first meeting, Christopher participated in the project as well; 
Jordan and Sanidhya were not present for any meetings and had no participation 
with the project at this point. The work was split into parts based on emulator 
cycles and instruction sets, with members being able to implement any part they 
wanted. The emulator was mainly programmed by Samuel and Claire (Siran), where 
Samuel did the implementation for the fetch and decode aspects, whilst Claire 
focused on the execute aspect and also helped out a bit with the decode. Work 
was also done to prepare for the assembler section, with utility functions being 
partially implemented by Christopher.

\section*{Group Efficiency}
Time zone differences due to remote working made communication difficult, but 
since work was separated into fairly disjoint parts and each part was able to be 
taken by anyone, provided they informed the group, overall progress was still 
made on the project.

\section*{Emulator Structure}
The fetch, decode and execute cycles are done in succession, with the ARM state 
memory location being passed into the execute function to be altered in the 
function. The decode function identifies the instruction type, and if it is a 
data processing instruction, the op code is identified instead. A data struct 
is used to store the instruction type and the relevant bits to be used for 
execution. Then, the instruction is passed into the relevant execute function, 
with each data processing op code having a separate one, and the other 
instruction types having collective functions. 

\section*{Potential Difficulties}
At this point, the emulator is not fully operational, but it does work for some 
instructions and progress has been made on the assembler section, so work has 
been eased for later sections once the emulator is completed. Also, there has 
been no discussion for the extension, partially due to increased individual 
workloads due to having non-participating members. - Christopher
\newline

When I tried to implement the excution part, since members in my group strated 
decode and execution part at the same time and we didn't discuss the structure 
of our codes before, I wasn't very clear about the structure of my codes, so I 
did part of the decode in execution part(eg. the decode\_data\_processing function) 
which caused difficulties later when we tried to implement pipeline function. 
Also when coding the function data\_processing\_instr, there are lots of 
duplicated codes which need to be improved further. And to make the file 
structure more elegant, the emulator.c file also needs to be separated into a 
few header files and sub-files. - Claire(Siran) Shen
\newline

For the first week of the project all in the group except myself and Claire were 
not present. This led to a very slow start in the project, but we decided to 
carry on anyway, so me and Claire(Siran) worked on the emulator, where I worked 
on the fetch and decode mainly whilst Claire mainly worked on the execute 
aspect. After the first group meeting Christopher showed up and started 
doing work on the assembler section which has been helpful. However the rest of
the group has done nothing in the slightest to help. 

Upon creation of the group I set up a whatsapp group chat so we could coordinate
easily, which I feel has helped us three with the implementation. However,
communication about how we will implement the emulator was not really talked
about which led to large amounts of clashes in code and style when we tried to 
merge them. I feel that now we are on the same page and this shows due to the 
progress we have made in the past week with only 3 people. But overall, I feel 
how we are working as a group of 3 now is close to optimal, only needing to add
a bit more communication between us.

Currently the Emulator is poorly structured since it is all in one file, this 
is hopefully going to be changed soon, to make it easier for navigation of the 
code. I feel this is very useful as a learning experience for the Assembler
section and hence we won't write all the code in one file. We will also use the
structure of how to decode the instructions in the assembler.

I feel the utility functions in the assembler may be tricky and laborious to
implement. However Christopher has already started implementing these to reduce
the amount of time it will take to program these.

-Samuel


\end{document}


